\documentclass[12pt]{article}

\usepackage[utf8]{inputenc}
% \usepackage{T1}{}
\usepackage[brazil]{babel}

\usepackage{graphicx}
\usepackage{url}
\usepackage{amsmath}

\author{Nome do autor aqui}
\title{Segundo exercício para o minicurso de \LaTeX{}}
\date{}

\begin{document}

\maketitle

\begin{abstract}
Esse é o segundo exercício do minicurso de \LaTeX{} ministrado na FACOM Techweek do ano de 2016. Tente fazer uma cópia deste documento utilizando a ferramenta \LaTeX{} fornecida no ambiente \url{www.sharelatex.com}.
\end{abstract}


\section{Introdução}
\label{sec:introducao}

No último exercício treinamos os comandos de estrutura básicos, listas, tabelas e figuras. Nesse exercício vamos treinar os ambientes matemáticos textual e gráfico. Vocês se lembram dos três tipos de ambientes matématicos apresentados em aula? Também vamos aprender como obter, armazenar e apresentar citações em nosso documento.

Qualquer dúvida basta me chamar e todo o conteúdo visto em aula pode ser encontrado no site \url{www.github.com/luizcoro/latex-minicurso}.

\section{As $3$ equações mais famosas}
\label{sec:sec2}

Nessa seção serão apresentadas as $3$ equações mais famosas na minha opnião. Para seleção dessas equações foi analisado a relevância para nossa disciplina. Algumas utilizam potências, outras utilizam elementos subscritos, outras frações etc. Todas as explicações serão extraídas (talvez integralmente) do \textit{website} \textit{Wikipedia}.

\subsection{Teorema de Pitágoras}

Em \cite{pitagoras} é mostrado que o teorema de Pitágoras é uma relação matemática entre os comprimentos dos lados de qualquer triângulo retângulo. Na geometria euclidiana, o teorema afirma que:

\begin{quote}
Em qualquer triângulo retângulo, o quadrado do comprimento da hipotenusa é igual à soma dos quadrados dos comprimentos dos catetos.
\end{quote}

A equação do teorema de Pitágoras é dada por:

$$c^2 = b^2 + a^2 \text{,}$$

onde $c$ representa o comprimento da hipotenusa, e $a$ e $b$ representam os comprimentos dos outros dois lados.

\subsection{Teorema fundamental do Cálculo}

O teorema fundamental do cálculo é a base das duas operações centrais do cálculo, diferenciação e integração, que são considerados como inversos um do outro. Isto significa que se uma função contínua é primeiramente integrada e depois diferenciada (ou vice-versa), volta-se na função original \cite{calculo}.

Formalmente, o teorema diz o seguinte:

Considere $f$ uma função contínua de valores reais, definida em um intervalo fechado $[a, b]$. Se $F$ for a função definida para $x$ em $[a, b]$ por

$$F(x) = \int_a^b f(t)dt$$

Se $F$ é uma função tal que $f(x) = F'(x)$ para todo $x$ em $[a, b]$, então:

$$\int_a^b f(x)dx = F(b) - F(a)$$


\subsection{Distribuição normal}

A distribuição normal é uma das mais importantes distribuições da estatística, conhecida também como Distribuição de Gauss ou Gaussiana. Além de descrever uma série de fenômenos físicos e financeiros, possui grande uso na estatística inferencial. É inteiramente descrita por seus parâmetros de média e desvio padrão, ou seja, conhecendo-se estes valores consegue-se determinar qualquer probabilidade em uma distribuição Normal \cite{normal}.

A função densidade de probabilidade da distribuição normal com média $\mu$ e a variância $\sigma^2$ (de forma equivalente, desvio padrão $\sigma$) é assim definida,


$$f(x,\mu,\sigma) = \frac{1}{\sqrt{2\pi\sigma}}e^{-\frac{(x-\mu)^2}{2\sigma^2}}$$

\subsection{Extra -- ambiente \texttt{cases}}

O \textit{Merge Sort}, ou ordenação por mistura, é um exemplo de algoritmo de ordenação do tipo dividir-para-conquistar.

Sua ideia básica consiste em Dividir (o problema em vários sub-problemas e resolver esses sub-problemas através da recursividade) e Conquistar (após todos os sub-problemas terem sido resolvidos ocorre a conquista que é a união das resoluções dos sub-problemas).

Os três passos úteis dos algoritmos dividir-para-conquistar, ou divide and conquer, que se aplicam ao merge sort são:

\begin{enumerate}
\item Dividir: Dividir os dados em subsequências pequenas;
\item Conquistar: Classificar as metades recursivamente aplicando o \textit{Merge Sort}; e
\item Combinar: Juntar as metades em um único conjunto já classificado.
\end{enumerate}

A recorrência para o custo $T(n)$, onde $n$ é o tamanho da entrada, é dada por:

$$T(n) =
\begin{cases}
    \Theta(1) & \text{se } x = 1 \\
    2T(\frac{n}{2}) + \Theta(n) & \text{se } x > 1
\end{cases}
$$
Resolvendo pelo Teorema Mestre percebe-se que essa recorrência tem a solução $T(n) = \Theta(nlog_2n)$.


\bibliography{referencias}
\bibliographystyle{plain}

\end{document}
